%\setlength{\parindent}{0.5in}
\begin{abstract}

    Neutrinos from fission nuclear reactors have been widely studied in particle and nuclear physics.
    In the last ten years, the antineutrino flux and spectrum was measured independently by short baseline reactor experiments. 
    Both flux and spectrum measurements showed discrepancies comparing to theoretical models based on historical measurements and nuclear databases.
    These discrepancies hint sterile neutrino oscillation in eV scale, as well as incomplete theoretical model. 
    PROSPECT, the Precision Reactor Oscillation and Spectrum experiment, was built to probe a sterile neutrino oscillation and precisely measure reactor antineutrino spectrum from a high $^{235}$U enrichment reactor.
    PROSPECT antineutrino detector is an optically segmented liquid scintillation detector deployed at 7~m to 9~m from the HFIR reactor at ORNL.
    To characterize the nonlinear detector energy response, a unique calibration and analysis strategy was made to overcome the typical challenge brought by particle multi-segment scatterings in PROSPECT detector.
    This dissertation details the analysis to calibrate the energy scale of PROSPECT antineutrino detector and PROSPECT's measurement antineutrino spectrum from $^{235}$U reactor.
\end{abstract}